% Rapport intermédiaire

% TOADD:
\textsl{
Note:\\
On adopte le préfixe PIC (pour PIlote de Capteur).
}

\section{Spécification des fonctions d'interface}

\subsection{Fonctions publiques}

\begin{m_desc}

 \item [PIC\_DrvInstall] \hfill\\
Installe le driver.
Le temps est réinitialisé, ainsi que la numérotation des messages.
Retourne 0 en cas de succès, -1 sinon.

 \item [PIC\_DrvRemove] \hfill\\
Désinstalle le driver.
Retourne 0 en cas de succès, -1 sinon.

 \item [PIC\_DevAdd] \hfill\\
 prend comme paramètres le nom du capteur et son identifiant associé (le choix de 
ces deux paramètres est laissé à l'utilisateur). Elle vérifie que aucun autre 
capteur n'utilise le même identifiant et/ou le même nom et ajoute le capteur
 si ces conditions sont vérifiées.
Le périphérique n'est ajouté que si le nombre de capteur utilisé après ajout
 est inférieur ou égal à 15. La fonction met également à jour le nombre 
de périphériques ajoutés.

Le nom et l'adresse des capteurs doivent être uniques.

Retourne 0 en cas de succès, -1 sinon.
ERRNO est placé à PIC\_E\_TOOMANYDEV si il y a trop de capteurs enregistrés.

 \item [PIC\_DevDelete] \hfill\\
supprime le périphérique demandé si il existe. 
Retourne 0 en cas de succès, -1 sinon.

\end{m_desc}


\section{Plan de test}

On se propose de pratiquer les tests suivants sur notre pilote de périphérique dans les situations suivantes :

\begin{enumerate}
 \item utilisation normale
 \item ajout de plus de quinze capteurs
 \item ouverture du même périphérique plus d'une fois
 \item utilisation de ioctl pour changer le fichier associé au périphérique
 \item ajout/suppression du périphérique, installation/désinstallation du pilote.
 \item test de charge du pilote (envoi de nombreux messages dans un intervalle bref)
\end{enumerate}
