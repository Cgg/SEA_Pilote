% Rapport

% TOADD:

\section{Spécification des fonctions d'interface}

\subsection{Spécifications générales}

Ce pilote est capable de gérer jusqu'à 15 capteurs connectés à la 
machine via un réseau de type CAN. Chacun dispose d'un identifiant
unique, codé sur 8 bits, et envoie un message composé de cet
identifiant ainsi que de 4 octets de données, exploitable par
l'application (ex. température).
Le driver se charge de dater l'arrivée de chaque message.
La datation est relative au driver, et n'a qu'un usage
de chronologie de réception des messages.
Il est possible d'accéder, par une lecture destructrice, aux
10 dernières valeurs envoyées par le capteur.

Il est possible, de façon transparante, de réassocier un descripteur
de fichier ouvert avec un nouveau capteur, si le précédent est 
par exemple défaillant.

\textsl{
Note:\\
On adopte le préfixe PIC (pour PIlote de Capteur).
}


\subsection{Fonctions publiques}

\begin{m_desc}

 \item [PIC\_DrvInstall()] \hfill\\
installe le driver.\\
Le temps est réinitialisé, ainsi que la numérotation des messages.\\
Retourne 0 en cas de succès, -1 sinon.\\
ERRNO est placé à PIC\_E\_PIC\_DEJA\_INSTALLE si le driver est déjà 
installé.

 \item [PIC\_DrvRemove()] \hfill\\
désinstalle le driver.\\
Les capteurs doivent être supprimés.\\
Retourne 0 en cas de succès, -1 sinon.\\
ERRNO est placé à PIC\_E\_PIC\_PAS\_INSTALLE si le driver n'est pas 
installé.

 \item [PIC\_DevAdd(char *const name, char const adresseCapteur)] \hfill\\
 prend comme paramètres le nom \textsl{name} du capteur et son identifiant \textsl{adresseCapteur} associé.\\
 Elle vérifie qu'aucun autre 
capteur n'utilise le même nom et ajoute le capteur
 si ces conditions sont vérifiées.\\
Le périphérique n'est ajouté que si le nombre de capteur utilisé après ajout
 est inférieur ou égal à 15.

Le nom et l'adresse des capteurs doivent être uniques.

Retourne le nombre de capteurs en cas de succès, -1 sinon.\\
ERRNO est placé à PIC\_E\_TOOMANYDEV si il y a trop de capteurs
 enregistrés, à PIC\_E\_PIC\_PAS\_INSTALLE si le driver n'est pas installé,
ou à PIC\_E\_DEV\_DEJA\_PRESENT si un capteur de ce nom éxiste.

 \item [PIC\_DevDelete(char * name)] \hfill\\
supprime le périphérique \textsl{name} si il existe.\\
Retourne 0 en cas de succès, -1 sinon.\\
ERRNO est positionné à PIC\_E\_DEV\_NON\_PRESENT si le capteur demandé 
n'est pas installé.

\item [PIC\_ioctl(int fd, int func, int arg)]\hfill\\
permet de contôler le driver.\\
Il est possible de réassocier le descripteur de fichier \textsl{fd} avec le capteur passé en \textsl{arg}.\\
\textsl{func} doit valoir PIC\_IOCTL\_FONCTION\_HOT\_SWAP, la seule fonction disponible.\\
Retourne 0 en cas de réussite, -1 en cas d'échec.\\
ERRNO est placé à PIC\_E\_PARAM\_INCORRECTS dans ce cas.

\item [open(char* name, int mode, int flag)]\hfill\\
ouvre le capteur \textsl{name} en lecture seule (on ne peut pas écrire
sur un capteur).
Le capteur à été ajouté avec \textsl{DevAdd} antérieurement.
 Ce capteur peut être ouvert par plusieurs utilisateurs.\\
Retourne le descripteur de fichier correspondant si l'appel réussi, -1 sinon.\\
ERRNO est positionné à PIC\_E\_DEV\_NON\_PRESENT si le capteur demandé 
n'est pas installé.

\item [close(int fd)]\hfill\\
ferme le descripteur de fichier \textsl{fd}.

\item [read(int fd, char* buffer, int nbBytes)]\hfill\\
lit de façon consommatrice la valeur la plus ancienne envoyée par le 
capteur dont le descripteur de fichier est  \textsl{fd}.\\
Une structure PIC\_MESSAGE\_CAPTEUR est copiée dans \textsl{buffer}.\\
\textsl{nbBytes} doit valoir PIC\_TAILLE\_MESSAGE\_TRAITE, sinon l'appel
retourne -1 et place ERRNO à PIC\_E\_PARAM\_INCORRECTS. 

Retourne PIC\_TAILLE\_MSG\_TRAITE en cas de succès, -1 sinon.\\
ERRNO est placé à PIC\_E\_PAS\_DE\_MESSAGE si la liste de message
est vide.


\item[Valeurs des codes ERRNO:]\hfill\\
\begin{center}
   \begin{tabular}{ | c | c | }
     \hline
Code                         &   Valeur\\ \hline
PIC\_E\_TOOMANYDEV		     &   ERRMAX +1\\ \hline
PIC\_E\_PARAM\_INCORRECTS    &   ERRMAX +1\\ \hline
PIC\_E\_PIC\_DEJA\_INSTALLE  &   ERRMAX +1\\ \hline
PIC\_E\_PIC\_PAS\_INSTALLE   &   ERRMAX +1\\ \hline
PIC\_E\_DEV\_DEJA\_PRESENT   &   ERRMAX +1\\ \hline
PIC\_E\_DEV\_NON\_PRESENT    &   ERRMAX +1\\ \hline
PIC\_E\_PAS\_DE\_MESSAGE	 &	 S\_objLib\_OBJ\_UNAVAILABLE\\ \hline
   \end{tabular}
 \end{center}

\end{m_desc}

\vfill
\pagebreak
